\documentclass{article}
\usepackage[utf8]{inputenc}

\input{compaq.tex}


\title{Elementos de probabilidad y estadística. Ayudantía 2.}
\date{9 de febrero de 2024}



\begin{document}

\maketitle


\begin{enumerate}
    \item (El problema del cumpleaños) Una clase tiene $n$ estudiantes. ¿Cuál es la
    probabilidad de que al menos dos tengan el mismo cumpleaños? (ignora los gemelos
    y los años bisiestos). Calcula la probabilidad para $n=10,15,22,30$ y 40.

    \item Un 14 de febrero en un lugar de Guanajuato, hay cinco parejas celebrando.
    Un operativo policiaco llega al lugar e inspecciona a cuatro personas al azar,
    ¿cuál es la probabilidad de que entre los inspeccionados haya al menos una pareja?

    \item P. R. de Montmort, escribió en 1708 un problema sobre el juego francés
    Jeu de Boules.  El objetivo del juego es  tirar pelotas de metal hacia una pelota 
    objetivo.  La pelota más cercana gana.  Supón que dos jugadores Irving y Samuel son igual
    de buenos en este juego. Irving lanza una pelota, y Samuel lanza dos. ¿Cuál es la
    probabilidad de que Irving gane?

    \item La NASA está desarrollando dos transbordadores espaciales ultra secretos,
    uno tiene dos motores y el otro tiene cuatro.  Todos los motores son idénticos 
    y tienen la misma probabilidad de fallar.  Cada uno está diseñado para volar si 
    al menos la mitad de sus motores funcionan.  José Luis, un científico visitante pregunta,
    ``El transbordador de cuatro motores es más confiable, ¿verdad?'' El técnico de
    la NASA responde que la probabilidad de fallo es secreta, pero que, de hecho, 
    ambos tienen la misma probabilidad de volar. El visitante dice entonces ``Entonces
    ahora sé cuál es la probabilidad de que un cohete falle y la probabilidad de que
    el transbordador vuele''. ¿Cómo lo descubrió y cuáles son estas probabilidades?
    

\end{enumerate}




\end{document}