\documentclass{article}
\usepackage[utf8]{inputenc}

\input{compaq.tex}

%\renewcommand{\P}[1]{P\left(#1\right)}

\title{Elementos de probabilidad y estadística. Ayudantía 1.}
\date{2 de febrero de 2024}



\begin{document}

\maketitle


\begin{enumerate}
    \item Sean $E, F, G$. Encuentra expresiones en términos de $E,F$ y $G$ 
    para los siguientes eventos,
    \begin{enumerate}
        \item sólo $F$ ocurre,
        \item tanto $E$ como $F$ ocurren, pero no $G$,
        \item por lo menos uno de los eventos ocurre,
        \item por lo menos dos eventos ocurren,
        \item los tres eventos ocurren,
        \item ningún evento ocurre,
        \item ocurre a lo sumo un evento,
        \item a lo sumo ocurren dos eventos.
    \end{enumerate}

    \item Una caja contiene tres canicas: una roja, una verde y una azul. Considera un
    experimento que consiste en tomar una canica de la caja, volverla a meter y tomar 
    otra canica de la caja. 
    
    \begin{enumerate}
        \item ¿Cuál es el espacio muestral? 
        \item Si en todo momento, cada canica tiene la misma probabilidad de ser 
        seleccionada, ¿cuál es la probabilidad de cada elemento en el espacio muestral?
    \end{enumerate}

    \item En el contexto del ejercicio anterior, ¿cómo cambian las respuestas cuando
    la primera canica que se toma no es devuelta a la caja?

    \item (Desigualdad de Bonferroni) Si $\P{E}=0.9$ y $\P{F}=0.8$, muestra que $\P{E\cap F}\ge 0.7$.
    En general, muestra que 
            
            \[ \P{E\cap F} \ge \P{E} + \P{F} - 1. \] 

    
    \item Prueba la siguiente desigualdad
    
            \[ \P{\bigcup_{i=1}^n A_i} \ge \sum_{i=1}^n \P{A_i} - n + 1 \]

    
    \item (Desigualdad de Boole) Muestra que
    
            \[ \P{\bigcup_{i=1}^n E_i} \le \sum_{i=1}^n \P{E_i}. \]

    \item Sean $E$ y $F$ dos eventos disjuntos ($E\cap F = \emptyset$) en el espacio 
    muestral de un experimento. Suponga que el experimento se repite hasta que ocurra
    alguno de $E$ o $F$. ¿Cuál es el espacio muestral de este nuevo experimento?

    \item Una moneda es lanzada hasta que aparecen dos caras seguidas. ¿Cuál es el 
    espacio muestral del experimento? Si la moneda es justa (la probabilidad de 
    ver una cara es $1/2$ al igual que la de ver un sol), ¿cuál es la probabilidad de
    que la moneda será lanzada exactamente cuatro veces?

    
\end{enumerate}




\end{document}