\documentclass{article}
\usepackage[utf8]{inputenc}

\input{compaq.tex}

\renewcommand{\P}[1]{P\left(#1\right)}

\title{Elementos de probabilidad y estadística. Ayudantía 3.}
\date{16 de Febrero de 2024}



\begin{document}

\maketitle

\begin{enumerate}
    \item Consideremos que tenemos seis pares de tazas y platos. Dos parejas (de tazas y platos) son rojas, dos son blancas y las dos restantes tienen impreso una estrella. Si las tazas son colocadas al azar encima de los platos, encuentre la probabilidad de que no haya una taza sobre un plato del mismo patrón.
    \item En una clase de 10 alumnos van a distribuirse 3 premios. Averiguar de cuántos modos
    puede hacerse si:

    \begin{enumerate}
        \item los premios son diferentes.
        \item los premios son iguales.
    \end{enumerate}

    (Notar que existen dos casos; que una persona pueda recibir más de un premio o que no pueda)

    \item Encuentra la cantidad total de diagonales de un polígono de $n$ lados.
    
    \item ¿Cuántos números de 4 dígitos se pueden formar con las cifras 1,2,\dots,9
    
    \begin{enumerate}
        \item Permitiendo repeticiones;
        \item Sin repeticiones;
        \item Si el último dígito ha de ser 1 y no se permiten repeticiones
    \end{enumerate}


    \item ¿Cuántos números de 3 cifras tienen todas sus cifras impares y al menos una de ellas está repetida?
    
    \item El emporio de pizza de Harry ofrece los siguientes diez ingredientes que se pueden agregar a la pizza de queso: pepperoni, champiñones, pimientos, salami, anchoas, camarón, tocino, cebolla, aceitunas y tomate. El cliente puede elegir cualquier combinación de ingredientes, incluyendo todos o ninguno. La tienda dice que ofrece más de mil pizzas distintas, ¿es verdad?
    
    
\end{enumerate}




\end{document}