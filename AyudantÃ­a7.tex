\documentclass{article}
\usepackage[utf8]{inputenc}

\input{compaq.tex}

\renewcommand{\P}[1]{P\left(#1\right)}

\title{Elementos de probabilidad y estadística. Ayudantía 5.}
\date{14 de marzo de 2024}



\begin{document}

\maketitle 

\begin{enumerate}

    \item Sea $X$ una variable aleatoria constante y $X \equiv 3/2$. Encuentra su función de masas de probabilidad función de distribución.

    \item Sea $U$ una variable aleatoria uniforme en $[0,1]$. Sea $X = U^2$. Encuentra la función de distribución de $X$.
    
    \item Sea $X$ una variable aleatoria con función de distribución 
    
    \[ F(x) = \left\{ \begin{array}{cc}
        0   & x < 0,\\
        x^2 & 0 \le x \le 1,\\
        1   & x > 1.
    \end{array} \right. \]

    Encuentra $\P{1/4 < X < 5}$, $\P{.2 < X < .8}$ y $\P{X = 1/2}$.

    \item Se lanza una moneda justa tres veces. Sea $X$ la variable aleatoria dada por el número de soles menos el número de águilas. Encuentra la función de distribución de $X$.
    
    \item ¿Cuáles de las siguientes son funciones de distribución?
    
    \begin{enumerate}
        \item $F(x) = \mathds{1}_{(2,\infty)}(x)$.
        \item $F(x) = \frac{x^2}{1 + x^2}$.
    \end{enumerate}

    \item Cinco hombres y cinco mujeres llegan aleatoriamente a un concierto y se forman en la taquilla, luego se les asignan números del 1 al 10 de acuerdo con su lugar en la fila. Sea $X$ la posición de la primera mujer en la fila. Encuentra la distribución de $X$.
    
    
    \item Sean $A$ y $B$ conjuntos y $f$ una función. Muestra que la imagen inversa $f^{-1}$ conserva operaciones de conjuntos, es decir $f^{-1}(A^c) = f^{-1}(A)^c$, $f^{-1}(A\cup B) = f^{-1}(A)\cup f^{-1}(B)$, $f^{-1}(A\cap B) = f^{-1}(A)\cap f^{-1}(B)$.
    
    \item Una función de distribución se llama puramente discontinua si es igual a la suma de sus saltos, es decir $F(x) = \sum_{y\le x}(F(y)-F(y-))$ para todo $x$.
    
    Muestra que cualquier función de distribución puede ser escrita como $F=aF_c + (1-a)F_d$, donde $F_c$ y $F_d$ son funciones de distribución continua y puramente discontinua, respectivamente, y $0\le a \le 1$.

    \item Demuestra lo siguiente.
    
    \begin{enumerate}
        \item Una función de distribución $F$ es puramente discontinua si y sólo si $\sum_y (F(y) - F(y-)) = 1$.
        \item La función de distribución de una variable aleatoria discreta es puramente discontinua.
    \end{enumerate}

    \item Sea $\{ X_j \}_{j\ge 1}$ una sucesión de variables aleatorias.
    
    \begin{enumerate}
        \item Muestra que $\sup_n X_n$ e $\inf_n X_n$ son variables aleatorias.
        \item Muestra que $\{\omega : \lim_{n\to\infty} X_n\text{ existe }\} \in \mathscr F$.
    \end{enumerate}

\end{enumerate}





\end{document}