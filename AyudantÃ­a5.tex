\documentclass{article}
\usepackage[utf8]{inputenc}

\input{compaq.tex}

\renewcommand{\P}[1]{P\left(#1\right)}

\title{Elementos de probabilidad y estadística. Ayudantía 5.}
\date{01 de marzo de 2024}



\begin{document}

\maketitle

Resuelve individualmente cada uno de los ejercicios usando los temas revisados en clase y argumentando cada paso. Al terminar, entrega tus soluciones al ayudante.


\begin{enumerate}

    \item Irving, Samuel y Esaul lanzan una moneda, si el resultado de una de las personas es distinto del de las otras dos, el juego termina y esta persona gana. Si los tres obtienen lo mismo, cada quien vuelve a lanzar su moneda. Suponiendo que la moneda es justa, ¿cuál es la probabilidad de que el juego termine con la primera ronda de lanzamientos? Si las tres monedas están sesgadas y tienen $1/4$ de probabilidad de resultar en cara, ¿cuál es la probabilidad de que el juego termine con la primera ronda?
    
    \item Supón que hay $n$ personas en una fiesta, todos usando sombrero. En un momento dado, todos tiran su sombrero al centro del salón y cada quien toma un sombrero al azar. Muestra que la probabilidad de que nadie se quede con su propio sombrero es
    
    \[ \frac{1}{2!} - \frac{1}{3!} + \frac1{4!} + \cdots + \frac{(-1)^n}{n!}. \]

    \textbf{Bonus:} ¿A dónde converge esta probabilidad cuando $n\to\infty$?

    \item Consideremos el juego del cubilete, i.e. 5 dados cuyas caras tienen las etiquetas 9,10,J,Q,K,A. Calcule las probabilidades (sólo deje las expresiones en términos de factoriales)
    \begin{enumerate}
        \item de obtener un par
        \item de obtener dos pares
        \item una tercia
        \item un full
        \item un poker
        \item una quintilla.
    \end{enumerate}

    \item En una clase de 10 alumnos van a distribuirse 3 premios. Averiguar de cuántos modos puede hacerse si:

    \begin{enumerate}
        \item los premios son diferentes.
        \item los premios son iguales.
    \end{enumerate}

    (Notar que existen dos casos; que una persona pueda recibir más de un premio o que no pueda)



    \item \textbf{Falsos positivos en un test de VIH}: Cierta prueba de anticuerpos para la infección de VIH se cree que es precisa por arriba del $99\%$. Esto significa que una prueba realizada será correcta con probabilidad de al menos 0.99, y estará equivocada con probabilidad no mayor a 0.01. 
    
    Hay dos tipos de error que pueden ocurrir en la prueba: puede indicar que una persona no infectada tiene VIH (falso positivo), o puede indicar que una persona infectada no tiene VIH (falso negativo). Supongamos que el error es de exactamente $1\%$ tanto para falsos positivos como para falsos negativos.
    
    La prevalencia del VIH en Canadá en 2005 fue de aproximadamente 2 infectados por cada 1000 personas en la población\footnote[1]{Screening for HIV: A Review of the Evidence for the U.S. Preventive Services Tsk Force, Chou et al., Annals of Internal Medicine, July 5, 2005, col. 143, no. 1, 55-73}. Una persona es elegida de manera aleatoria de la población. Dicha persona es analizada con la prueba y esta resulta positiva. ¿Cuál es la probabilidad de que esta persona en realidad NO esté infectada?
\end{enumerate}





\end{document}